\documentclass[UTF8]{ctexart}
\ctexset { section = { format={\Large \bfseries } } }
\pagestyle{plain}
\usepackage{float}
\usepackage{amsmath}
\usepackage{amssymb}
\usepackage{listings}
\usepackage{graphicx}%插入图片宏包
\usepackage{xcolor}
\usepackage{geometry}
\geometry{a4paper,scale=0.8}
\usepackage{caption}
\usepackage{subcaption}
\usepackage[colorlinks=true, linkcolor=blue, citecolor=blue, urlcolor=blue]{hyperref}
\captionsetup[figure]{name={Figure}}
\captionsetup[table]{name={Table}}
\definecolor{Rhodamine}{RGB}{227,11,92}


\lstset{
language=Python, % 设置语言
basicstyle=\ttfamily\small, % 设置字体族
breaklines=true, % 自动换行
keywordstyle=\bfseries\color{blue}, % 设置关键字为粗体,
morekeywords={}, % 设置更多的关键字,用逗号分隔
emph={self}, % 指定强调词,如果有多个,用逗号隔开
emphstyle=\bfseries\color{Rhodamine}, % 强调词样式设置
commentstyle=\color{black!50!white}, % 设置注释样式,斜体,浅灰色
stringstyle=\bfseries\color{red!90!black}, % 设置字符串样式
columns=flexible,
numbers=left, % 显示行号在左边
numbersep=2em, % 设置行号的具体位置
numberstyle=\footnotesize, % 缩小行号
frame=single, % 边框
framesep=1em % 设置代码与边框的距离
}

\title{\textbf{Image Processing Homework 6}}
\author{吴嘉骜 21307130203}
\date{\today}

\begin{document}

\maketitle

\noindent
\section{}
\setlength{\parindent}{0pt}
实现K类均值分类的分割算法或基于高斯混合模型的分割算法(备注:二选一就行,不可以调用别的库实现的函数),并使用噪声污染过的图像(如P=0.1%的椒盐噪声)测试一下算法:

(1)测试二类分割,并对比自己实现的算法的分割结果与阈值算法(如OSTU或基于最大熵)二值化的结果;

(2)测试多类(大于等于三类)分割(请自己设定分割标签类别的个数);

(3)针对噪声图像,讨论为什么分割的结果不准确,有什么方法可以取得更好的分割结果(备注:不要求实现该方法,只是讨论)。


\textbf{Solution}:\\
The code from \texttt{freqfilter.py} is long and thus shown in the \hyperlink{code1}{Appendix}.\\


For Gaussian noise, we directly generate normal distribution random numbers by \texttt{np.random.normal}.\\
For Rayleigh noise, we generate random numbers by the inverse transform method. Note that the cumulative distribution function of Rayleigh distribution is
$F(z;a,b)=1-e^{-\frac{(z-a)^2}{2b^2}}$, where $a$ is the location parameter and $b$ is the scale parameter. Let $u$ be a uniform random variable in $[0,1]$, then $F^{-1}(u;a,b)=a+b\sqrt{-2\ln(1-u)}$.\\
Thus, we can generate Rayleigh distribution random numbers by applying $F^{-1}$ to uniform random numbers.\\
For salt-and-pepper noise, we determine the polluting probability by comparing $u$ with a preset threshold $p$.\\

\textbf{Analysis}:\\


\section{}
请使用课程学习的形态学操作实现二值图像的补洞和离散点去除。形态学操作功能要自己代码实现,不能调用库。如下左图是有问题图像,右图是目标图像。
\textbf{Solution}:\\


\newpage
\appendix
\hypertarget{code1}{\section{Code for Problem 1}}
\begin{lstlisting}
from PIL import Image
import numpy as np
\end{lstlisting}

\end{document}